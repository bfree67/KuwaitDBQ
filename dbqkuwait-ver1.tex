\documentclass[preprint,12pt,a4paper,authoryear]{elsarticle}
%\usepackage[dvipdfm]{graphicx} 
\makeatletter
\def\ps@pprintTitle{%
 \let\@oddhead\@empty
 \let\@evenhead\@empty
 \def\@oddfoot{}%
 \let\@evenfoot\@oddfoot}
\makeatother

\usepackage{graphicx}
%% The amssymb package provides various useful mathematical symbols
\usepackage{lineno}
%% The lineno packages adds line numbers. Start line numbering with
%% \begin{linenumbers}, end it with \end{linenumbers}. Or switch it on
%% for the whole article with \linenumbers after \end{frontmatter}.

\usepackage{float}
\usepackage{amsmath}
%for tables using merged columns
\usepackage{multirow}
\usepackage{booktabs}

\usepackage{doi}
\usepackage{url}

%\usepackage{biblatex}
%\addbibresource{dbq.bib}

%\journal{Transportation Journal}

\title{The Driver Behaviour Questionnaire in Kuwait}

\begin{document}

\maketitle

%\begin{linenumbers}
\begin{frontmatter}

%%%%%%%%%%%%%%%%%%%%%%%%%%%%%%%%%%%%%%%%%%%%
\author[add1]{Jamal Ahmad Al Matawah \corref{cor1}}
\author[add2]{Other guy}
\author[add3]{Brian Freeman}

\cortext[cor1]{Corresponding author (jamaln1@hotmail.com)}

\address[add1]{Public Authority for Applied Education and Training, Dept of Civil Engineering, Kuwait}
\address[add2]{Jordan}
\address[add3]{School of Engineering, University of Guelph, Guelph, Ontario, N1G 2W1, Canada}

%%%%%%%%%%%%%%%%%%%%%%%%%%%%%%%%%%%%%

\begin{abstract}
%% Text of abstract
The Manchester Driver Behaviour Questionnaire (DBQ) is widely used to measure driving styles and investigate the relationship between driving behaviour and accident involvement. Recent evaluations of different population groups have taken place throughout the world, including countries in the Arabian Gulf. This study seeks to extend the application of the DBQ to Kuwait and its mix of native and expatriate drivers, by examining the relationships between DBQ factors and accident involvement.  In our study, 536 respondents (425  Kuwaiti and 111 Non-Kuwaitis) completed paper survey questionnaires based on the DBQ parameters as well as background information. The results showed that young Kuwaiti male drivers scored higher in almost all areas. Factor analysis resulted in four significant violation areas (speed related  errors, anger related  errors, speed related lapses, and anger related lapses) errors, and lapses. However, there were a number of differences in the factor structure when compared to the theoretical four-factor structure of the DBQ.  Regression analyses showed that errors, lapses, and aggression-speeding violations predicted accident involvement based on demographic variables (age, sex, and annual mileage).
\\

\end{abstract}

\begin{keyword}
DBQ \sep Kuwait \sep driver behavior \sep traffic safety \sep aggression speeding 
\end{keyword}

\end{frontmatter}
 
\section{Introduction}
Driving related accidents result in over 500 fatalities per year in Kuwait (KUNA, 2016) and represents the third largest cause of death in this small country \citep{Wang2016}. Accidents and accident likelihood have been studied by many researchers with root causes attributed to individual driving styles and driving habits.  The Manchester Driver Behaviour Questionnaire (DBQ) was developed to measure individual concepts and types of driver behaviour and has been used many researchers in many countries including Britain \citep{Reason1990}, Qatar and the United Arab Emirates \citep{Bener2008}, Canada \citep{Cordazzo2014}, Denmark \citep{Martinussen2013}, France \citep{Gu2014}, Finland and the Netherlands \citep{Lajunen2004}, Australia \citep{Stephens2016}, and Turkey \citep{Sumer2003}. 

The DBQ measures how often drivers experience three categories defined as lapses, errors and violations. For lapses, the questionnaire asked how often drivers try to pull away from traffic lights in third gear, how often they operated the wrong switch, took the wrong lane approaching roundabout or junction, misread signs on exiting roundabouts, how often they felt disorientated, reached a wrong destination, forgot where they had left their car in the car park or hit something when reversing. Lapses are usually considered not to be life-threatening and were more commonly reported by females rather than by male drivers. Age was also found to be statistically associated with lapses, with older drivers tending to report more. 

Errors were defined by Reason (1990) as constituting a failure of planned action and include failures in observation and misjudgements. For errors, respondents were asked how often they failed to see a `Stop' or `Give Way' sign and narrowly avoid colliding with right of way traffic, how often they failed to observe cyclists, pedestrians crossing side roads, failed to check the rear-view mirror before pulling out or changing lanes, and failed to pay attention to the vehicle in front when it was about to turn off the main road. Other errors identified were braking too quickly on a slippery road, or steering the wrong way in a skid, underestimating the speed of an oncoming vehicle when overtaking, and attempting to overtake someone signalling an offside turn. \citep{Reason1990}. Reason (1990) defined violations as  actions that were deliberate deviations from practices considered to be important to maintain safety in a potentially hazardous environment. Questions included how often drivers disregarded the speed limits late at night or very early in the morning, crossed a junction knowing that the traffic lights were changing, drove close to the car in front in an aggressive manner, overtook on the inside, raced with other drivers, showed hostility to a class of road user, or expressed anger verbally. Violations are typical of aggressive behaviour driving. \citep{Reason1990}. 

Based on DBQ results, driver violation scores were found to be a much better predictor of level of accident involvement than the error or lapse score. However, according to Reason et al. (1990), both errors and violations potentially lead to accidents, whereas lapses are unlikely to have a major impact on driving safety.    

Many road safety professionals cite speeding and alcohol as the most important immediate precursors of crashes. The DBQ analyses divide aberrant driving behaviour into three kinds: speeding, drink-driving and other general classes of violations, with speeding being the most frequent violation. 

Lawton et al. (1997) categorised violations according to motivational interpersonal aggression (aggressive violation) and deliberate deviation (ordinary violation). On the other hand, Lajunen and Parker (2001) and Lajunen et al. (1998) stated that violation items are sometimes difficult to differentiate, because of local conditions, snow on the road (Scandinavia) or larger number of cyclists (Holland). Also, culture plays a part. Sounding the horn clearly reflects aggression in Scandinavia, while in Southern Europe, the horn is used more liberally. Culturally sensitive items need careful consideration for international comparisons. Therefore, the distinction between ordinary violations and aggressive violations may depend on the context and the intention behind the act.   

The literature referred above noted variations in the categories of lapses, errors and violations that reflect true cultural differences. Traffic cultures may vary at regional level. The DBQ item "brake too quickly on slippery road" has very different meanings in countries with a long snowy winter and in countries where snow tyres are never required. Similarly, cycling is more common as a daily form of transport in the Netherlands than in the hilly parts of Turkey, and so attention to cyclists is much more relevant in the former country. Traffic environment and culture play a major role. For example, a striking difference in Muslim culture is that alcohol is not commonly consumed.  Additionally in Gulf states, the percentage of expatriate residents is often very high compared to citizen drivers. In Kuwait, the percentage of non-Kuwaitis is almost 70\% of a total population of 4.4 million in 2016 \citep{PACI2017}. The ratio of cars to drivers is also high, with at least 12\% more cars than licensed drivers Kuwait, as shown in Table \ref{tb:residents}.

\begin{table}[H]
\centering
\caption{Licensed drivers and registered vehicles in Kuwait from 2013-2015 (CSB, 2017)}
\label{tb:drivers}
\begin{tabular}{@{}cccc@{}}
\toprule
\textbf{Year} & \textbf{Licensed Drivers} & \textbf{Registered Vehicles} & \textbf{ratio} \\ \midrule
2013 & 1,497,605 & 1,748,424 & 1.17 \\
2014 & 1,641,793 & 1,837,372 & 1.12 \\
2015 & 1,686,138 & 1,925,168 & 1.14 \\ \bottomrule
\end{tabular}
\end{table}

Each year, an additional 78,000 vehicles are added to Kuwait's road networks, with over 64,100 operated by private drivers as shown as the slopes in Figure \ref{fig:regvehicles}. Using the Registered vehicles ratio of 1.14 from Table \ref{tb:drivers}, the actual number of added vehicles per year can be assumed to be 69,000 total vehicles, of which 56,000 are private drivers.

 %
\begin{figure}[H]
\centering
\includegraphics[width=\textwidth,keepaspectratio]{images/vehicles}  %assumes jpg extension
\caption{Total and privately registered vehicles in Kuwait (CSB, 2017).}
\label{fig:regvehicles}
\end{figure}
%

Aggressive driving behaviour has various kinds of definitions. The most comprehensive definition is: "The operation of a motor vehicle in a manner that endangers or is likely to endanger people or property".  The factors listed by James and Nahl (2000) as representative of aggressive driving are: Running stop signals, Blocking intersections, Failing to yield right-of-way, Weaving in/out of traffic, Speeding above the limit, Tailgating, Failure to use indicators when required, Changing speed erratically, Blocking other vehicles, Communicating threats or insults with voice, Gestures, or sounding the horn unnecessarily, Intentionally breaking suddenly, and chasing other vehicles  \citep{James2000}. Other studies added additional factors, such as careless driving, failure to stop for pedestrians, and cell phone usage  \citep{Stephens2016}.

\section{Material and methods}

\subsection{Description of the questionnaire survey data contents}
 The questionnaire consisted of various variables which were grouped into six sections (parts). The first section included general information and consisted of general demographic information such age, sex, nationality, residential area, occupation, living place, education level, and marital status. The second section included Driving Characteristics and consisted of variables related to the vehicle use in terms of distance travelled annually (kilometres), driving experience, seatbelt usage, speeding, vehicle type and activities whilst driving. The third section took into account accident history and requested information related to accidents involved in, accident causes, and types of injuries suffered (if any). The fourth section asked about Driver Behaviour and consisted of variables related to driving patterns and individual actions. These variables are categorised as Violations (10 questions), Errors (8 questions) and Lapses (8 questions). 
\\
These 26 questions were further classified according to various aspects, such as lack of attention, speeding, overtaking, ignoring priorities, passing red traffic lights, not leaving enough distance from the vehicle in front, and carelessness while driving. The responses to the questions were taken on a six-point Likert Scale ranging from 0 (Never) to 5 (Nearly all the time). 

The fifth section, Driving Strengths and Weakness, consisted of 20 questions related to the behaviour of dangerous driving situations and reactions. The responses to these questions were also taken with a five-point Likert Scale ranging from 0 (Definitely weak) to 4 (Definitely strong). 

The sixth section, Road Safety Strategies, consisted of variables related to remedial measures such as road design, enforcement of traffic laws, and road safety campaigns. The last page of the questionnaire provided free space for suggestions and comments. \\

\subsection{Questionnaire survey data collection}

A pilot survey of 50 questionnaires was distributed to drivers in Kuwaiti in November 2016 to identify potential problems of design. In the main survey, from 3 December 2016 to 15 May 2017, 700 questionnaires were distributed to a random sample of drivers at various locations in Kuwait.  Of the returned surveys, 164  were rejected, either because they were incomplete or because answers were considered to be unrealistic, giving a total number of respondents of 536 and an overall response rate of 76\%. 
        
\subsection{Characteristics of respondents}

A breakdown of the survey respondents is shown in Table \ref{tb:respondents}.  The overwhelming majority were male and Kuwaiti. Over half held a bachelors degree or higher.  The mean age of the participants was 31.8, with the youngest being 18 and the oldest 72. The response rate from Kuwaiti drivers was higher than from non-Kuwaiti drivers (79.5\% and 20.5\%, respectively). 

\begin{table}[H]
\centering
\caption{Background characteristics of respondents.}
\label{tb:respondents}
\begin{tabular}{@{}lc@{}}
\toprule
\textbf{Sex} &  \\ \midrule
Male & 74.0\% \\
Female & 26.0\% \\ \midrule
\textbf{Education} &  \\ \midrule
High school or below & 17.9\% \\
Diploma & 27.2\% \\
Bachelors degree & 42.4\% \\
Postgraduate & 12.5\% \\ \midrule
\textbf{Martial status} &  \\ \midrule
Married & 49.1\% \\
Unmarried & 50.9\% \\ \midrule
\textbf{Citizenship} &  \\ \midrule
Kuwaiti & 79.5\% \\
non-Kuwaiti & 20.5\% \\ \bottomrule
\end{tabular}
\end{table}

 Based on the respondent demographics in Table \ref{tb:respondents}, the survey sample did not represent the age distribution in the 18-24 years old range, or Kuwaiti/non-Kuwaiti ratio of the overall population.  Figure \ref{fig:differences} shows the relative differences of the sample cohort versus the population of Kuwait. Most non-Kuwaitis do not drive, especially family members and low-waged staff that makes up the majority of the Kuwaiti expatriate work force. This discrepancy was accepted by other studies conducted in the region \citep{Bener2008}. 
 
 %
\begin{figure}[H]
\centering
\includegraphics[width=\textwidth,keepaspectratio]{images/ages}  %assumes jpg extension
\caption{Discrepancies of demographics for (a) Kuwait vs. non-Kuwaiti drivers and (b) age group 18-24.}
\label{fig:differences}
\end{figure}
%
The age discrepancy for the 18-24 years old range is driven by the access to large number respondents at the college and represents the largest population subgroup (29.8\%)  of Kuwaiti drivers. An assumption was made to keep all responses from this age group, as compared to randomly rejecting responses in order to bring the proportions in line with the actual proportions of 18-24 years old in Kuwait, as this is one of the main target groups impacted by traffic safety. Additionally, the extra respondents in the one category would not impact the analytical statistics as the degrees of freedom, n, normalized results.

In Table \ref{tab:drive-exp}, over 59\% of the respondents had more than 5 years of driving experience.

\begin{table}[H]
\centering
\caption{Driving Experience of Respondents}
\label{tab:drive-exp}
\begin{tabular}{@{}ccc@{}}
\toprule
\textbf{Years of driving} & \textbf{Frequency} & \textbf{Percentage} \\ \midrule
\textless 2 & 60 & 11.2\% \\
2 - 5 & 156 & 29.1\% \\
5 - 10 & 93 & 17.4\% \\
\textgreater10 & 227 & 42.3\% \\
Total & 536 & 100.0\% \\ \bottomrule
\end{tabular}
\end{table}

Annual kilometres ranged from $<$5,000 km to $>$ 40,000 km. Table \ref{tab:annualkm} and Figure Figure 1 shows the distribution of the annual average kilometres for men and women separately: 21,768 km for men and 13,964 km for females (on average 19,730 km). This shows that men are driving around 61\% more than the distance that females are driving.

\begin{table}[H]
\centering
\caption{Frequency distribution of average km/year driven.}
\label{tab:annualkm}
\begin{tabular}{@{}ccc@{}}
\toprule
\textbf{km/year} & \textbf{Frequency} & \textbf{Percent} \\ \midrule
\textless5,000 & 33 & 6.2\% \\
5,000-10,000 & 101 & 18.8\% \\
10,000-15,000 & 112 & 20.9\% \\
15,000-20,000 & 56 & 10.4\% \\
20,000-25,000 & 63 & 11.8\% \\
25,000-30,000 & 56 & 10.4\% \\
30,000-35,000 & 35 & 6.5\% \\
35,000-40,000 & 28 & 5.2\% \\
\textgreater40,000 & 52 & 9.7\% \\
Total & 536 & 100\% \\ \bottomrule
\end{tabular}
\end{table}

Figure \ref{fig:avekmgender} shows the average driven km/yr for men and women, with men reporting they drive over 61\% more than the women do.
%
\begin{figure}[H]
\centering
\includegraphics[width=\textwidth,keepaspectratio]{images/pic1}  %assumes jpg extension
\caption{Average km/yr based on gender.}
\label{fig:avekmgender}
\end{figure}
%

\subsubsection{Accidents}
From the 536 respondents, 71\% reported having been involved in one or more accidents during their driving time, and 29\% had not. For those involved in previous accidents, 80.3\%  said the accidents involved property damage only (PDO), while 19.4\% reported their accidents caused injuries. Only one respondent reported a fatal accident. Using this data, we estimated a fatal injury ratio of 1 death per 74 injuries. The types of accidents are summarized in Table \ref{tab:accidents}.

\begin{table}[H]
\centering
\caption{Types of accidents reported by respondents.}
\label{tab:accidents}
\begin{tabular}{@{}lcc@{}}
\toprule
\textbf{Accident type} & \textbf{Frequency} & \textbf{Percent} \\ \midrule
PDO & 305 & 80.3\% \\
Injury & 74 & 19.4\% \\
Fatality & 1 & 0.3\% \\
Total & 380 & 100\% \\ \bottomrule
\end{tabular}
\end{table}

\subsubsection{Accident causation}
The respondents were asked about the causes of accidents (either their own fault or another driver's). The results in Table \ref{tab:causes} show that speed represents 33.9\% of accident causes and careless driving represents 17.1\%. General traffic violations and other combinations contributed 35.8\% and 11.8\% respectively.  Accidents involving alcohol were only 1.3\% in the survey.  It should be noted that Kuwait is a dry country that prohibits the import and sale of alcoholic beverages \citep{FCO2013}. 

\begin{table}[H]
\centering
\caption{Accident causes}
\label{tab:causes}
\begin{tabular}{@{}lcc@{}}
\toprule
\textbf{Accident factors} & \textbf{Frequency} & \textbf{Percent} \\ \midrule
Violations & 130 & 35.8\% \\
Speeding & 123 & 33.9\% \\
Carelessness & 62 & 17.1\% \\
Others & 43 & 11.8\% \\
Alcohol involved & 5 & 1.4\% \\ \bottomrule
\end{tabular}
\end{table}

\subsubsection{Violations}
The respondents were asked if they obtained right-light, speeding, or parking tickets. Speeding and parking were the most common violations in Kuwait \citep{CSB2017}. The results in Table \ref{tab:violations} show that speeding was the most frequent type of violation.

\begin{table}[H]
\centering
\caption{Reported violations}
\label{tab:violations}
\begin{tabular}{@{}lcc@{}}
\toprule
\textbf{Violation} & \textbf{Frequency} & \textbf{Percent} \\ \midrule
Over Speed & 157 & 37.6\% \\
Parking & 129 & 30.9\% \\
Red-light & 44 & 10.5\% \\
Parking and Over Speed & 28 & 6.7\% \\
Red-light and Over Speed & 26 & 6.2\% \\
Red-light, parking and Over Speed & 24 & 5.7\% \\
Red-light and parking & 7 & 1.7\% \\
Others & 3 & 0.7\% \\ \bottomrule
\end{tabular}
\end{table}

\subsubsection{Activities while driving}
The respondents were asked about what activities that they engage in while driving. The results in Table \ref{tab:activities} show that respondents often use their mobile phones while driving, despite being illegal to do so in Kuwait \citep{MOI2014}.

\begin{table}[H]
\centering
\caption{Driving activities}
\label{tab:activities}
\begin{tabular}{@{}lcc@{}}
\toprule
\textbf{Activity} & \textbf{Frequency} & \textbf{Percent} \\ \midrule
Using mobile phone & 229 & 55.6\% \\
Other activities & 83 & 20.1\% \\
Smoking & 45 & 10.9\% \\
Children in front & 34 & 8.3\% \\
Folding legs & 16 & 3.9\% \\
Reading newspaper & 3 & 0.7\% \\
Drinking alcohol & 2 & 0.5\% \\ \bottomrule
\end{tabular}
\end{table}

\subsubsection{Seat belt usage}
Only 40.7\% of the respondents said they regularly use seat belts while 43.7\% said they occasionally used them. A small group reported, 15.7\% , reported not using seat belts at all. The reason behind of those using seat belts regularly  and those who use them occasionally  are shown in Table \ref{tab:excuses}. 

\begin{table}[H]
\centering
\caption{Reasons why seat belts are not regularly used.}
\label{tab:excuses}
\begin{tabular}{@{}lcc@{}}
\toprule
\textbf{Reason} & \textbf{Frequency} & \textbf{Percent} \\ \midrule
Discomfort & 90 & 28.9\% \\
Forget to use & 79 & 25.4\% \\
Inconvenience & 53 & 17.0\% \\
Combination of reasons & 26 & 8.4\% \\
Fear of being trapped & 19 & 6.1\% \\
Other reasons & 18 & 5.8\% \\
Interfered with clothes & 15 & 4.8\% \\
Not Necessary & 11 & 3.5\% \\ \bottomrule
\end{tabular}
\end{table}


\section{Results of the Kuwait DBQ}

\subsection{Reliability Analysis}
A lower bound reliability estimate was computed for each category (Violations, Errors and Lapses) separately using Cronbach's $\alpha$,  a commonly used statistic for estimating reliability of test scores \citep{Warrens2014}. The $\alpha$ of each category was calculated using SPSS software and represents the average covariance between item-pairs and variance of the total score and given by

\begin{equation}
\label{eq:alpha}
\alpha = \frac{N*\bar{c}}{\bar{\sigma^{2}}+(N-1)*\bar{c})}
\end{equation}

\noindent
where $N$ is the number of item-pair being compared, $bar{c}$ is the average covariance between item pairs, and $\bar{\sigma^{2}}$ is the average variance. Using Cronbach's $\alpha$ as a reliability indicator test is common within literature, but studies show that it should not be a measure of internal consistency \citep{Sijtsma2009, Tavokol2011}.

\begin{table}[H]
\centering
\caption{Reliability analysis using Cronbach's $\alpha$.}
\label{tab:alpha}
\begin{tabular}{@{}cc@{}}
\toprule
\textbf{Item-pair} & \textbf{$\alpha$} \\ \midrule
Violation & 0.867 \\
Errors & 0.817 \\
Lapses & 0.847 \\ \bottomrule
\end{tabular}
\end{table} 

Results in Table \ref{tab:alpha} show a good range of reliability between variables. Scores $>$ 0.9 may reflect redundant or duplicate questions \citep{Streiner2003}.

\subsection{Principal Component Analysis}
The dimensionality of the 10 items in the Violations category was analysed using principal component analysis (PCA).   Two factors were rotated using a varimax rotation procedure. The rotated solution yielded two interpretable factors: speed related violations and anger related violations. Each of the two factors consist of 5 items. Two items, V6 and V10, were in both the anger related factors and speed related factor. As a result, these two items were moved to the speed related factor. The final factors consist of 7 items in the speed related factor (V1, V2, V3, V4, V6, V7, and V10)  and 3 items in the anger related factor (V5, V8, and V9). 

PCA was also used to find factors of 8 items related with the Errors category and 8 items of the Lapses category. In both cases, the rotated solution yielded only one interpretable factor.

\subsection{Overall speed related behaviour score with other factors}

The overall speed related score was introduced as a dependent variable with other factors (Age, Gender, Education Level, Marital Status, Nationality, and Driver Experience) as an independent variable. An overall speed related score (SRS) was calculated as an average of the 7 questions for each individual driver (V1, V2, V3, V4, V6, V7, and V10). The t-test was used to compare significant differences in overall speed related scores between two independent groups (such as Gender, Marital Status, and Nationality). When the analysis involved three or more groups, such as Age, Education Level and Driver Experience, the one-way ANOVA technique was used. In both the t-test and one-way ANOVA, the level of significance was set at the 95\% confidence interval level (p$<$ 0.05). 

\subsubsection{Age versus SRS for violations }

A one-way ANOVA was conducted to evaluate the relationship between age and the SRS. The hypothesis assumed that the younger age group drivers have more aggressive driving behaviour on the road than the older age group drivers. The independent variable tested, age, had five age ranges: 18-24, 25-29, 30-39, 40-49 \& 50-above. The dependent variable was the speed related score. A higher speed related score indicated more aggressive behaviour on the road. The ANOVA showed statistical significance.

Follow-up tests were conducted to evaluate pair wise differences among the means. The \textit{post hoc} comparisons were conducted using Dunnett's C test. The results of these tests are given in Table \ref{tab:experience}. 

There were significant differences in the means between the age groups (18-24/25-29), (18-24/30-39), (18-24/40-49), (18-24/50-above), (25-29/40-49), (25-29/50-above), (30-39/50-above),  but no significant differences in the means between the age groups(25-29/30-39) (30-39/ 40-49). 

The drivers of the age group 18-24 showed the highest speed related behaviour ($\bar{x}$ = 2.48), whereas the drivers of the age group 50-above showed the lowest aggressive behaviour ($\bar{x}$ = 0.88) as shown in Table \ref{tab:experience}.

\begin{table}[H]
\centering
\caption{ANOVA results for SRS.}
\label{tab:experience}
\resizebox{\columnwidth}{!}{%
\begin{tabular}{@{}lccccccccc@{}}
\toprule
\textbf{Age Group} & \textbf{N} & \textbf{$\bar{x}$} & \textbf{SD} & \textbf{18-24} & \textbf{25-29} & \textbf{30-39} & \textbf{40-49} & \textbf{F} & \textbf{p} \\ \midrule
18-24 & 225 & 2.48 & 0.89 &  &  &  &  & 55.465 & $<1E-3$ \\
25-29 & 72 & 1.82 & 1.02 & * &  &  &  &  &  \\
30-39 & 92 & 1.57 & 1.11 & * & NS &  &  &  &  \\
40-49 & 85 & 1.29 & 0.76 & * & * & NS &  &  &  \\
50-above & 62 & 0.88 & 0.61 & * & * & * & * &  &  \\  \midrule
\textbf{Education level} & \textbf{N} & \textbf{$\bar{x}$} & \textbf{SD} & \textbf{Up to High school} & \textbf{Diploma} & \textbf{Bachelor} & \textbf{} & \textbf{F} & \textbf{p} \\  \midrule
Up to High school & 96 & 2.11 & 1.15 &  &  &  &  & 20.136 & $<1E-3$ \\
Diploma & 146 & 2.21 & 0.98 & NS &  &  &  &  &  \\
Bachelor & 227 & 1.74 & 1.04 & * & * &  &  &  &  \\
Postgraduate & 67 & 1.13 & 0.81 & * & * & * &  &  &  \\  \midrule
\textbf{Groups} & \textbf{N} & \textbf{M} & \textbf{SD} & \textbf{Less than 2 years} & \textbf{2-5 years} & \textbf{5-10 years} & \textbf{} & \textbf{F} & \textbf{p} \\  \midrule
Less than 2 years & 60 & 2.12 & 0.92 &  &  &  &  & 43.118 & $<1E-3$ \\
2-5 years & 156 & 2.22 & 0.94 & NS &  &  &  &  &  \\
5-10 years & 93 & 1.74 & 1.11 & NS & * &  &  &  &  \\
More than 10 years & 227 & 1.14 & 0.94 & * & * & * &  &  &  \\ \bottomrule
* - Significant (p$<$0.05) & & & & & & & & \\
NS - Not Significant (p $\geq$ 0.05) & & & & & & & & \\
\end{tabular}
} %end resize
\end{table}

The results of the one-way ANOVA supported the hypothesis that the younger age group displays more aggressive driving on the road than the older age group. Young drivers, compared with other groups, are more likely to underestimate the probability of specific risks caused by traffic situations Brown \& Gorger, 1988; Deery, 1999) and they overestimate their own driving skills (Moe, 1986).

\subsubsection{Gender versus SRS}

A t-test was conducted to evaluate the hypothesis that male drivers have more aggressive driving behaviour on the road than female drivers.  The results of the t-test are shown in Table \ref{tab:sexdiff} and support the hypothesis that male drivers on average have more aggressive driving behaviour than female drivers. Our results are similar to the results presented by Laapotti et al. (2003), which evaluated driver attitudes towards road safety in Finland \citep{Laapotti2003}. Their study revealed that female drivers also had a more positive attitude towards road safety and rules than male drivers.

\begin{table}[H]
\centering
\caption{Results of t-tests for SRS.}
\label{tab:sexdiff}
\begin{tabular}{@{}lccccc@{}}
\toprule
\textbf{Gender} & \textbf{n} & \textbf{$\bar{x}$} & \textbf{SD} & \textbf{t} & \textbf{Result} \\ \midrule
Male & 396 & 1.9402 & 1.09899 & 3.015 & * \\
Female & 140 & 1.6393 & .98366 &  &  \\ \midrule
\textbf{Nationality} & \textbf{n} & \textbf{$\bar{x}$} & \textbf{SD} & \textbf{t} & \textbf{Result} \\ \midrule
Kuwait & 426 & 2.0556 & 1.03820 & 8.77 & * \\
Non-Kuwaiti & 110 & 1.1100 & .88130 &  &  \\ \midrule
\textbf{marital status} & \textbf{n} & \textbf{$\bar{x}$} & \textbf{SD} & \textbf{t} & \textbf{Result} \\ \midrule
Single & 263 & 2.277 & 0.96431 & 9.466 &* \\
Married & 273 & 1.461 & 1.029 &  &  \\ \midrule
\textbf{Prior accident} & \textbf{n} & \textbf{$\bar{x}$} & \textbf{SD} & \textbf{t} & \textbf{Result} \\ \midrule
Yes & 380 & 1.9516 & 1.07849 & 3.082 & * \\
No & 156 & 1.6423 & 1.04579 &  &  \\ \bottomrule
* - Significant (p$<$0.05) & & & & &  \\
\end{tabular}
\end{table}

\subsubsection{Nationality versus SRS}

A t-test was conducted to evaluate the hypothesis that Kuwaiti drivers have more aggressive driving behaviour than non-Kuwaiti drivers. The results supported the hypothesis that Kuwaiti drivers are more aggressive drivers than non-Kuwaiti drivers) as shown in Table \ref{tab:sexdiff}. 

One reason to explain this result is that there are more young Kuwaiti drivers than young non-Kuwaiti drivers in the country's population as shown in Figure \ref{fig:differences}. In order to account for this difference, the  t-test was run again after randomly excluding 225 cases of young drivers in the age range 18-24. Nonetheless, Kuwaiti drivers still tested significantly for more aggressive driving behaviour than non-Kuwaiti drivers.

Several possible reasons exist that explain why non-Kuwaitis are less aggressive while driving Kuwait. Expatriates must have a driver's license from their home country, which may have more rigorous requirements. Expatriate drivers are often professional drivers and responsible for fines if they receive them. Expatriate drivers also tend to have lower financial status than Kuwaitis and therefore try to avoid unnecessary fines and expenses. Expatriate drivers tend to be older and educated if they are not professional drivers. Kuwaiti law requires a bachelor's degree for non-Kuwaiti private drivers \citep{KUNA2014}. Lastly, expatriate drivers often do not have the same influence, or \textit{ wasta}, within the traffic department that allows many Kuwaitis to avoid paying fines. The practical implication of the last explanation is that fines on Kuwaiti citizens are ineffective. 

\subsubsection{Marital status versus SRS}

An independent-sample t-test was conducted to evaluate the hypothesis that drivers who were single had more aggressive driving behaviour than drivers who were married. The results support the hypothesis that single drivers on average had more aggressive driving behaviour than married drivers as shown in Table \ref{tab:sexdiff}. There were more young single drivers than married drivers in the sample (89\% of drivers in the age group 18-24 were single). Married drivers are assumed to have more concerns, possibly due to family responsibilities. There is an implied need for further education and training of young, unmarried drivers, either through the media or driving schools in order to increase their feeling of responsibility and improve their hazard perception. 

\subsubsection{Prior accident versus SRS}

An independent-sample t-test was conducted to evaluate the hypothesis that drivers who were involved with one or more accidents, had more aggressive driving behaviour than drivers who were not involved in accidents.  The results of this test supported the hypothesis that drivers with one or more accident showed more aggressive behaviour compared to drivers with no accidents.

\subsection{Overall anger related behaviour score (violation) with other factors}

The overall anger related score (ARS) was introduced as the dependent variable, with other factors (Age, Gender, Education Level, Marital Status, Nationality, and Driver Experience) acting as independent variables. An ARS was calculated as the average of the 3-questions for each individual driver (V5, V8, V9).  ANOVA was used to compare driver ages, education levels, and years of experience. The results in Table \ref{tab:anger} shows significant differences from these tests. The t-test was used to compare the significant differences in overall ARS between two independent groups (such as Gender, Marital Status, and Nationality). A higher ARS indicates more aggressive behaviour on the road.

\subsubsection{Age versus ARS}

There were significant differences in the means between the age groups, (18-24/30-39), (18-24/40-49), (18-24/50-above), (25-29/50-above), but no significant differences in the means between the age groups (18-24/25-29), (25-29/30-39), (25-29/40-49), (30-39/40-49),  (30-39/50-above) and (40-49/50-above) were found. The drivers of the age group 18-24 showed the highest anger related behaviour, whereas the drivers of the age group 50-above showed the lowest aggressive behaviour as shown in Table \ref{tab:anger}.

\subsubsection{Education level versus ARS}

Comparing the ARS to education levels, the ANOVA results were also significant, but the only significant difference was between the postgraduate and the other education level groups. Drivers with high school level educations showed the highest anger related behaviour, whereas the drivers of the postgraduate group showed the lowest aggressive behaviour as shown in Table \ref{tab:anger}.

\subsubsection{Driving experience versus ARS}
The ANOVA results were also significant between the ARS and the years of driving. The only significant differences were between drivers with more than 10 years driving experience drivers with less experience. The drivers with of 2-5 years of experience showed the highest anger related behaviour, whereas drivers with more than 10 years of experience showed the lowest aggressive behaviour as shown in Table \ref{tab:anger}.

\begin{table}[H]
\centering
\caption{ANOVA results for ARS.}
\label{tab:anger}
\resizebox{\columnwidth}{!}{%
\begin{tabular}{@{}lccccccccc@{}}
\toprule
\textbf{Age Group} & \textbf{N} & \textbf{$\bar{x}$} & \textbf{SD} & \textbf{18-24} & \textbf{25-29} & \textbf{30-39} & \textbf{40-49} & \textbf{F} & \textbf{p} \\ \midrule
18-24 & 225 & 2.49 & 1.12 &  &  &  &  & 19.35 & 0.000 \\
25-29 & 72 & 2.07 & 1.40 & NS &  &  &  &  &  \\
30-39 & 92 & 1.76 & 1.07 & * & NS &  &  &  &  \\
40-49 & 85 & 1.63 & 0.913 & * & NS & NS &  &  &  \\
50-above & 62 & 1.38 & 0.870 & * & * & NS & NS &  &  \\ \midrule
\textbf{Education level} & \textbf{N} & \textbf{$\bar{x}$} & \textbf{SD} & \textbf{Up to High school} & \textbf{Diploma} & \textbf{Bachelor} & \textbf{} & \textbf{F} & \textbf{p} \\ \midrule
Up to High school & 96 & 2.32 & 1.25 &  &  &  &  & 9.24 & 0.000 \\
Diploma & 146 & 2.24 & 1.098 & NS &  &  &  &  &  \\
Bachelor & 227 & 1.97 & 1.19 & NS & NS &  &  &  &  \\
Postgraduate & 67 & 1.46 & 0.87 & * & * & * &  &  &  \\ \midrule
\textbf{Groups} & \textbf{N} & \textbf{M} & \textbf{SD} & \textbf{Less than 2 years} & \textbf{2-5 years} & \textbf{5-10 years} & \textbf{} & \textbf{F} & \textbf{p} \\ \midrule
\textless 2 years & 60 & 2.36 & 1.10 &  &  &  &  & 20.48 & 0.000 \\
2-5 years & 156 & 2.46 & 1.09 & NS &  &  &  &  &  \\
5-10 years & 93 & 2.19 & 1.39 & NS & NS &  &  &  &  \\
\textgreater10 years & 227 & 1.62 & 0.99 & * & * & * &  &  &  \\ \bottomrule
* - Significant (p$<$0.05) & & & & & & & & & \\
NS - Not Significant (p $\geq$ 0.05) & & & & & & & &  &\\
\end{tabular}
} %end resize
\end{table}

\subsubsection{Gender versus ARS}

An independent-sample t-test was conducted to evaluate if there was a significant differences between male and female drivers in terms of ARS. The results in Table \ref{tab:married} test showed that there was not a significant difference between males and females. 

\subsubsection{Nationality versus ARS}

An independent-sample t-test was conducted to evaluate the hypothesis that Kuwaiti drivers have more aggressive driving behaviour than non-Kuwaiti drivers in terms of anger related score. The test showed a significant difference between Kuwaiti and non-Kuwaiti as shown in Table \ref{tab:married}.

\subsubsection{Marital status versus ARS}

An independent-sample t-test was conducted to evaluate the hypothesis that single drivers have more aggressive driving behaviour than married  drivers in terms of anger related score. The test showed a significant difference between single and married drivers as shown in Table \ref{tab:married}.

\subsubsection{Prior accidents versus ARS}

An independent-sample t-test was conducted to evaluate if there was a significant differences between whether a drivers was involved in a prior accident in terms of ARS. The results in Table \ref{tab:married} test showed that there was not a significant difference between respondents who had been in an accident and those that were not. 

\begin{table}[H]
\centering
\caption{Results of t-tests for ARS.}
\label{tab:married}
\resizebox{\columnwidth}{!}{%
\begin{tabular}{@{}lccccc@{}}
\toprule
\textbf{Gender} & \textbf{n} & \textbf{$\bar{x}$} & \textbf{SD} & \textbf{t} & \textbf{Result} \\ \midrule
Male & 396 & 2.03 & 1.138 & -0.57 & NS \\
Female & 140 & 2.095 & 1.261 &  &  \\ \midrule
\textbf{Nationality} & \textbf{n} & \textbf{$\bar{x}$} & \textbf{SD} & \textbf{t} & \textbf{Result} \\ \midrule
Kuwaiti & 426 & 2.19 & 1.161 & 6.129 & * \\
Non-Kuwaiti & 110 & 1.49 & 1.034 &  &  \\ \midrule
\textbf{Marital status} & \textbf{n} & \textbf{$\bar{x}$} & \textbf{SD} & \textbf{t} & \textbf{Result} \\ \midrule
Single & 263 & 2.356 & 1.154 & 6.212 & * \\
Married & 273 & 1.748 & 1.109 &  &  \\ \midrule
\textbf{Prior accident} & \textbf{n} & \textbf{$\bar{x}$} & \textbf{SD} & \textbf{t} & \textbf{Result} \\ \midrule
Yes & 380 & 2.096 & 1.151 & 1.51 & NS \\
No & 156 & 1.925 & 1.209 &  &  \\ \bottomrule
* - Significant (p$<$0.05) & & & & &  \\
NS - Not Significant (p $\geq$ 0.05) & & & & & \\
\end{tabular}
} %end resize
\end{table}

\subsection{Results of Error scores}

An ANOVA test was run on where the overall Errors score (ES) was the dependent variables against age of the driver, education level and years of driving experience. These results are shown in Table \ref{tab:errors}. Independent t-tests comparing Errors against gender, nationality, marital status and involvement in prior accidents are shown in Table \ref{tab:errgender}. 

\subsubsection{Age groups and Errors}

There were significant differences in the means between age groups (18-24/30-39), (18-24/40-49),  (18-24/50-above), (25-29/50-above) (30-39/50-above), and (40-49/50-above), but no significant differences in the means between age groups (18-24/25-29) , (25-29,/30-39), (25-29/40-49), and (30-39/40-49). The drivers of the age group 18-24 showed the highest ES, whereas the drivers 50-above showed the lowest ES as shown in Table \ref{tab:errors}.

\subsubsection{Education level and ES}

When comparing education level, the ANOVA results were significant, but the only significant differences were between the postgraduate and the other education level groups. Drivers with high school educations and below showed the highest anger related behaviour, whereas drivers with postgraduate group showed the lowest aggressive behaviour as shown in Table \ref{tab:errors}.

\subsubsection{Experience and ES}

While the ANOVA was significant between the Errors score and driving experience, the only significant differences were between drivers with more than 10 years of driving experience and those with less 5 or less years of experience. The driving experience of less than 2 years showed the highest ES, whereas  drivers of more than 10 years showed the lowest ES behaviour  as shown in Table \ref{tab:errors}.

\begin{table}[H]
\centering
\caption{ANOVA results for ES.}
\label{tab:errors}
\resizebox{\columnwidth}{!}{%
\begin{tabular}{@{}lccccccccc@{}}
\toprule
\textbf{Age Group} & \textbf{n} & \textbf{$\bar{x}$} & \textbf{SD} & \textbf{18-24} & \textbf{25-29} & \textbf{30-39} & \textbf{40-49} & \textbf{F} & \textbf{p} \\ \midrule
18-24 & 225 & 1.46 & 0.905 &  &  &  &  & 10.03 & 0.000 \\
25-29 & 72 & 1.24 & 0.993 & NS &  &  &  &  &  \\
30-39 & 92 & 1.076 & 0.771 & * & NS &  &  &  &  \\
40-49 & 85 & 1.18 & 0.877 & * & NS & NS &  &  &  \\
50-above & 62 & 0.75 & 0.605 & * & * & * & * &  &  \\ \midrule
\textbf{Education level} & \textbf{N} & \textbf{$\bar{x}$} & \textbf{SD} & \textbf{Up to High school} & \textbf{Diploma} & \textbf{Bachelor} & \textbf{} & \textbf{F} & \textbf{p} \\ \midrule
Up to High school & 96 & 1.44 & 0.868 &  &  &  &  & 4.094 & 0.007 \\
Diploma & 146 & 1.39 & 0.802 & NS &  &  &  &  &  \\
Bachelor & 227 & 1.27 & 0.975 & NS & NS &  &  &  &  \\
Postgraduate & 67 & 0.96 & 0.734 & * & * & * &  &  &  \\ \midrule
\textbf{Groups} & \textbf{N} & \textbf{M} & \textbf{SD} & \textbf{\textless 2 years} & \textbf{2-5 years} & \textbf{5-10 years} & \textbf{} & \textbf{F} & \textbf{p} \\ \midrule
\textless 2 years & 60 & 1.49 & 0.84 &  &  &  &  & 6.9 & 0.000 \\
2-5 years & 156 & 1.39 & 0.9 & NS &  &  &  &  &  \\
5-10 years & 93 & 1.37 & 1.04 & NS & NS &  &  &  &  \\
\textgreater 10 years & 227 & 1.06 & 0.8 & * & * & NS &  &  &  \\ \bottomrule
* - Significant (p$<$0.05) & & & & & & & & & \\
NS - Not Significant (p $\geq$ 0.05) & & & & & & & &  &\\
\end{tabular}
} %end resize
\end{table}

\subsubsection{Gender and ES}
An independent-sample t-test was conducted to evaluate if there was a significant differences between male and female drivers in terms of ES. The test showed no significant difference between males and female as shown in Table \ref{tab:errgender}.

\subsubsection{Nationality and ES }
An independent-sample t-test was conducted to evaluate if there was a significant differences between Kuwaiti and non-Kuwaiti drivers in terms of ES. The test was not significant difference between Kuwaiti and non-Kuwaiti as shown in Table \ref{tab:errgender}.

\subsubsection{Marital status and ES}
An independent-sample t-test was conducted to evaluate the hypothesis that single drivers more errors than married drivers. There was a significant difference between single and married drivers as shown in Table \ref{tab:errgender}.

\begin{table}[H]
\centering
\caption{Results of t-tests for ES.}
\label{tab:errgender}
\resizebox{\columnwidth}{!}{%
\begin{tabular}{@{}lccccc@{}}
\toprule
\textbf{Gender} & \textbf{n} & \textbf{$\bar{x}$} & \textbf{SD} & \textbf{t} & \textbf{Result} \\ \midrule
Male & 396 & 1.24 & 0.85 & -0.708 & NS \\
Female & 140 & 1.31 & 0.99 &  &  \\  \midrule
\textbf{Nationality} & \textbf{n} & \textbf{$\bar{x}$} & \textbf{SD} & \textbf{t} &\textbf{Result} \\  \midrule
Kuwaiti & 426 & 1.29 & 0.87 & 1.775 & NS \\
Non-Kuwaiti & 110 & 1.11 & 0.92 &  &  \\  \midrule
\textbf{Marital status} & \textbf{n} & \textbf{$\bar{x}$} & \textbf{SD} & \textbf{t} & \textbf{Result} \\  \midrule
Single & 263 & 1.37 & 0.92 & 2.910 & * \\
Married & 273 & 1.14 & 0.85 &  &  \\  \midrule
\textbf{Accident involved} & \textbf{n} & \textbf{$\bar{x}$} & \textbf{SD} & \textbf{t} &\textbf{Result} \\  \midrule
Yes & 380 & 1.26 & 0.82 & -0.019 &NS \\
No & 156 & 1.25 & 1.03 &  &  \\ \bottomrule
* - Significant (p$<$0.05) & & & & &  \\
NS - Not Significant (p $\geq$ 0.05) & & & & &  \\
\end{tabular}
} %end resize
\end{table}

\subsection{Overall Lapses score with other factors}
An ANOVA test was run where the Lapse Score (LS) was the dependent variables against age of the driver, education level and years of driving experience. These results are shown in Table \ref{tab:educationlapses}. Independent t-tests comparing LS against gender, nationality, marital status and involvement in prior accidents are shown in Table \ref{tab:lapses}.

\subsubsection{Age groups and LS}

There were significant differences in the means between the age groups (18-24/40-50), (18-24/50-above), (25-29/50-above), and (30-39/50-above),  but no significant differences in the means between the age groups (18-24/25-29), (18-24/ 30-39), (25-29/30-39), (25-29/40-49), (30-39/40-49), and (40-49/50-above) were found. The drivers of the age group 25-29 showed the highest LS, whereas the drivers of the age group 50-above showed the lowest LS as shown in Table \ref{tab:educationlapses}.

\subsubsection{Education level and LS}

The ANOVA for education level was only significant between the postgraduate and the other education level groups. Drivers with a high school education or less had the highest LS, where drivers with postgraduate education had the lowest LS as shown in Table \ref{tab:educationlapses}.

\subsubsection{Experience and LS}

The ANOVA was only significant between drivers with more than 10 years driving experience and drivers with less experience. Drivers with less than 2 years experience showed the highest LS where drivers with more than 10 years of experience showed the lowest LS as shown in Table \ref{tab:educationlapses}.

\begin{table}[H]
\centering
\caption{ANOVA results for LS.}
\label{tab:educationlapses}
\resizebox{\columnwidth}{!}{%
\begin{tabular}{@{}lccccccccc@{}}
\toprule
\textbf{Age Group} & \textbf{n} & \textbf{$\bar{x}$} & \textbf{SD} & \textbf{18-24} & \textbf{25-29} & \textbf{30-39} & \textbf{40-49} & \textbf{F} & \textbf{p} \\ \midrule
18-24 & 225 & 1.84 & 1 &  &  &  &  & 8.223 & 0.000 \\
25-29 & 72 & 1.89 & 1.1 & NS &  &  &  &  &  \\
30-39 & 92 & 1.57 & 1 & NS & NS &  &  &  &  \\
40-49 & 85 & 1.46 & 0.87 & * & NS & NS &  &  &  \\
50-above & 62 & 1.14 & 0.7 & * & * & * & NS &  &  \\ \midrule
\textbf{Education level} & \textbf{n} & \textbf{$\bar{x}$} & \textbf{SD} & \textbf{Up to High school} & \textbf{Diploma} & \textbf{Bachelor} & \textbf{} & \textbf{F} & \textbf{p} \\ \midrule
Up to High school & 96 & 1.75 & 1.03 &  &  &  &  & 3.125 & 0. 026 \\
Diploma & 146 & 1.7 & 0.99 & NS &  &  &  &  &  \\
Bachelor & 227 & 1.7 & 1.04 & NS & NS &  &  &  &  \\
Postgraduate & 67 & 1.31 & 0.84 & * & * & * &  &  &  \\ \midrule
\textbf{Groups} & \textbf{n} & \textbf{M} & \textbf{SD} & \textbf{Less than 2 years} & \textbf{2-5 years} & \textbf{5-10 years} & \textbf{} & \textbf{F} & \textbf{p} \\ \midrule
\textless 2 years & 60 & 1.94 & 1.02 &  &  &  &  & 7.117 & 0.000 \\
2-5 years & 156 & 1.75 & 0.99 & NS &  &  &  &  &  \\
5-10 years & 93 & 1.86 & 1.08 & NS & NS &  &  &  &  \\
\textgreater 10 years & 227 & 1.43 & 0.93 & * & * & * &  &  &  \\ \bottomrule
* - Significant (p$<$0.05) & & & & & & & & \\
NS - Not Significant (p $\geq$ 0.05) & & & & & & & & \\
\end{tabular}
} %end resize
\end{table}

\subsubsection{Gender and LS}
An independent-sample t-test was conducted to evaluate the differences between male and female drivers in terms of LS. There was a significant difference between men and women. Women showed higher LS then men as shown in Table \ref{tab:lapses}.

\subsubsection{Nationality and Lapses}
An independent-sample t-test was conducted to evaluate differences between Kuwaiti and non-Kuwaiti drivers in terms of LS. There was a significant difference between Kuwaitis and non-Kuwaitis as shown in Table \ref{tab:lapses}.

\subsubsection{Marital Statues and Lapses}
An independent-sample t-test was conducted to evaluate difference between single drivers than married drivers. The test showed a significant difference between single drivers and married drivers as shown in Table \ref{tab:lapses}.

\begin{table}[H]
\centering
\caption{Results of t-tests for LS.}
\label{tab:lapses}
\resizebox{\columnwidth}{!}{%
\begin{tabular}{@{}lccccc@{}}
\toprule
\textbf{Gender} & \textbf{n} & \textbf{$\bar{x}$} & \textbf{SD} & \textbf{t} & \textbf{Result} \\ \midrule
Male & 396 & 1.56 & 0.933 & -3.848 & * \\
Female & 140 & 1.93 & 1.144 &  &  \\ \midrule
\textbf{Nationality} & \textbf{n} & \textbf{$\bar{x}$} & \textbf{SD} & \textbf{t} & \textbf{Result} \\ \midrule
Kuwaiti & 426 & 1.71 & 1.01 & 2.709 & * \\
Non-Kuwaiti & 110 & 1.43 & 0.94 &  &  \\ \midrule
\textbf{marital status} & \textbf{n} & \textbf{$\bar{x}$} & \textbf{SD} & \textbf{t} &\textbf{Result} \\ \midrule
Single & 263 & 1.77 & 1.04 & 2.66 & * \\
Married & 273 & 1.54 & 0.96 &  &  \\ \midrule
\textbf{Accident involved} & \textbf{n} & \textbf{$\bar{x}$} & \textbf{SD} & \textbf{t} &\textbf{Result} \\ \midrule
Yes & 380 & 1.65 & 0.95 & -0.207 &NS \\
No & 156 & 1.67 & 1.11 &  &  \\ \bottomrule
* - Significant (p$<$0.05) & & & & &  \\
NS - Not Significant (p $\geq$ 0.05) & & & & &  \\
\end{tabular}
} %end resize
\end{table}

\subsection{Regression Analysis}

A multiple regression analysis was conducted to evaluate how the various predictor variables represent the overall speed related violations index conducted by the drivers. The four predictors used in the regression analysis were AGE-GROUP, Nationality (KU-NONKU), GENDER and traffic-accident-involved (ACCIDENT-INVO). The linear combination of the predictors' variables was significant in relation to Violations.

The sample multiple correlation coefficients were 0.595, indicating that approximately 35\% variance of the speed related violations index in the sample as shown in Table  \ref{tab:regression}.

\begin{table}[H]
\centering
\caption{Regression coefficients}
\label{tab:regression}
\begin{tabular}{@{}cccc@{}}
\toprule
\textbf{R} & \textbf{R$^{2}$} & \textbf{Adjusted  R$^{2}$} & \textbf{SEE} \\ \midrule
0,595 & 0.354 & 0.350 & 0.868 \\ \bottomrule
\end{tabular}
\end{table}

The prediction equation for the standardized variables can be represented as 

\begin{equation}
SR= 4.111C -0.354A – 0.43G -0.505N -0.189AI
\end{equation}

\noindent
and can be accounted for by the linear combination of the four predictor variables where $SR$ is the violation score (speed related),  $A$ is age, $G$ is Gender, $N$ is Nationality and $AI$ is Accident Involvement and $C$ is a constant as shown in Table \ref{tab:regression-coef}.

\begin{table}[H]
\centering
\caption{Coefficients for regression model.}
\label{tab:regression-coef}
\begin{tabular}{@{}lccccc@{}}
\toprule
\textbf{} & \multicolumn{2}{c}{\textbf{Unstandardized Coefficients}} & \multicolumn{3}{c}{\textbf{Standardized Coefficients}} \\ 
\textbf{Coefficients} & \textbf{B} & \textbf{Std. Errors} & \textbf{Beta} & \textbf{t} & \textbf{p} \\ \midrule
C & 4.111 & 0.191 &  & 21.524 & 0 \\
A & -0.354 & 0.028 & -0. 476 & -12.687 & 0 \\
G & – 0.43 & 0.086 & -0.175 & -4.969 & 0 \\
N & -0.505 & 0.102 & -0.189 & -4.927 & 0 \\
AI & -0.189 & 0.081 & -0.08 & -2.24 & 0.025 \\ \bottomrule
\end{tabular}
\end{table}

The correlation between each predictor and the criterion variable is shown in Table \ref{tab:corr}.

\begin{table}[H]
\centering
\caption{Correlation between each predictor and criterion variable.}
\label{tab:corr}
\begin{tabular}{@{}cc@{}}
\toprule
\textbf{Predictor} & \textbf{Corr} \\ \midrule
Age-Group & - 0.523** \\
Kuwaiti/non-Kuwaiti & - 0.355** \\
Gender & - 0.123** \\
Accidents-Involved & - 0.131* \\ \bottomrule
\end{tabular}
\end{table}

\section{Conclusions}
Based on our results age, has the most impact on driver behaviours. Young male drivers who don't continue past high school appear to be more aggressive and more likely to have an accident than other subgroups. This result is consistent with other studies that identify younger drivers as accident risks \citep{dewinters2010}. This is an alarming situation requiring more driver education and training then currently available in Kuwait. Enforcement is an issue that is also a concern. Speed and violations play important roles in accident occurrence, especially for young Kuwaiti drivers, who appear to be more aggressive in driving, since they do not pay much attention to enforcement. 

In the case of aggressive driver behaviour, age and nationality are also main contributing factors. Driving experience, gender, and marital status are important factors in behaviour, but they are clearly related to age. Having specified the factors that affect behaviour and accidents, focusing on dealing with these factors can help to improve driver behaviour and road safety in Kuwait.  

Additional research is required identify supporting factors contributing to aggressive driving. Future work should consider the amount of driver education received, number of times a driver test was taken before passing, nationalities of non-Kuwaitis, and time living in Kuwait.

\section{Acknowledgements}
This research was partially funded by the Public Authority for Applied Education and Technology. We are grateful for the support received from the Kuwait Traffic Safety Society in helping distributing the survey.
  
\section{References}

%\end{linenumbers}
%%%%%%%%%%%%%%%%%%%%%%%%%%%%%%%%%%%%%%%%%%%%%%%%%%%%%%% end 
\bibliography{dbqbib}{}
%\bibliographystyle{chicago}
\bibliographystyle{abbrvnat}

%\printbibliography
\end{document}
